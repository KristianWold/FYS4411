%\abstractintoc % Add abstract to Table of Contents 
%\abstractnum   % Format abstract like a chapter
                % Remove if abstract should not be on its own page

% \makeatletter
% \renewenvironment{abstract}{%
%     \if@twocolumn
%       \section*{\abstractname}%
%     \else %% <- here I've removed \small
%       \begin{center}%
%         {\bfseries \large\abstractname\vspace{\z@}}%  %% <- here I've added \Large
%       \end{center}%
%       \quotation
%     \fi}
%     {\if@twocolumn\else\endquotation\fi}
% \makeatother

\begin{abstract}
A novel neural network architecture has been built to obtain uncorrelated samples from the position wave function in finite potentials. In this paper, we study bosons in a harmonic oscillator in one, two and three dimensions, both in the interacting and non-interacting case. In the interacting case, a shielded Coulomb potential on the form
\begin{equation*}
	V(\vec{r_1}, \vec{r_2}) = \frac{\alpha}{\sqrt{|\vec{r_1} - \vec{r_2}|^2 + \beta^2}}
\end{equation*}    
was used as interaction.
For two and six non-interacting bosons, the ground state energy was calculated to a precision of about $0.1\%$ with respect to the analytical result for all number of dimensions . 
For two interacting bosons, the ground state energy was calculated to a precision of about $1\%$ with respect to the DMC and CISD calculations used as benchmarks for all number of dimensions.

The one body densities were estimated to some accuracy, but with some notable flaws. In the 1D case for two interacting bosons, the one-body density displayed the characteristic seperation of two peaks, although not as sharply as the benchmark. The one body density also suffered from a slight asymmetry. In higher dimensions, the radial one body density varied slightly from the benchmark, for both the interacting and non-interacting case.   




\end{abstract}
