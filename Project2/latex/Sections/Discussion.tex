\section{Results and Discussion}\label{sec:Discussion}

If not otherwise specified, the results have been derived using the following parameters:
\begin{itemize}
	\item Harmonic oscillator frequency $\omega = 1$
	\item Coulomb interaction strength and shielding constants $\alpha = 1$, $\beta = 0$
	\item Metropolis step length $1$
	\item Number of Metropolis steps $\textsf{steps} = 20$
	\item Batch size of 500
	\item Training duration of 500 epochs
	\item Two layer DNN architecture with $32$ nodes each  
\end{itemize}


\subsection{Pure DNN Model for One Particle In Harmonic Oscillator}
\subsubsection{One Particle in 1D Harmonic Oscillator, ReLU vs tanh}

\begin{figure}[H]
	\includegraphics[]{figures/one_part_wavefunc.pdf}
	\caption{Wave function of particle in 1D harmonic oscillator, approximated using DNN with tanh and ReLU activations, respectively. The Metropolis step length was set to $2$. The results are plotted against the analytical result}
	\label{fig:one_part_func}
\end{figure}

To examine the performance and behavior of wave function Ansatz by DNN, we begin with
a simple system of one particle in a 1D harmonic oscillator. Both \(\tanh\) and
ReLU activation functions were used to compare their performance. The
Metropolis step length was set to \(2\) to yield a \(50\%\) acceptance rate. The
resulting wave function is shown in~\cref{fig:one_part_training1}.
Upon first inspection, the trial wave function with ReLU activation 
fails spectacularly in approximating the analytical result, while
$\tanh$ produces a very close-lying approximation.


\begin{figure}[H]
	\includegraphics[]{figures/one_part_training1.pdf}
	\caption{Energy estimated from each batch during training of the $\tanh$ and $ReLU$ model on one particle in 1D harmonic oscillator}
	\label{fig:one_part_training1}
\end{figure}

The minimization of $\langle E \rangle$ during training for ReLU and $\tanh$
can be seen in \autoref{fig:one_part_training1}, and reveals even more
serious problems with ReLU. While the energy of the $\tanh$ model smoothly
decreases towards the analytical value of $E=0.5$, the energy of the ReLU
model varies wildly, even undercutting the analytical value. This is disastrous,
as it violates the variational principle.


\begin{figure}[H]
	\includegraphics[]{figures/one_part_local_kinetic.pdf}
	\caption{Local kinetic energy of the $\tanh$ and ReLU model trained on one particle in $1D$ harmonic oscillator}
	\label{fig:one_part_local_kinetic}
\end{figure} 

\autoref{fig:one_part_local_kinetic} shows the kinetic term of the local energy
as a function of position, plotted plotted
against the analytical result. The local kinetic energy of ReLU
model is ill-behaved, characterized by many discrete steps instead of smooth
curve. This is likely the cause of its problematic behavior. 
Since the local kinetic energy relies on the Laplacian of
the trial wave function, the use of activation functions with discontinuous
derivatives, such as ReLU, appears to produce models which cannot approximate
wave functions with appropriate curvature. 

Further, \autoref{fig:one_part_local_kinetic} shows an interesting feature of
the $\tanh$ model. While it closely approximates the correct local kinetic
energy in the center part, it fails for positions far from origo. A possible
explanation is that since samples are using the Metropolis
algorithm, configurations yielding a small value of the wave function
will be sampled less often. As a result, the model may struggle to
learn the correct approximation of the kinetic term for these areas. However,
this is not a problem for numerical accuracy, since the same configurations
that the model struggle to learn will seldom be sampled, and hence their
contribution to the expectation values will be negligible.

As the ReLU fails to produce physical results, it is not used in the rest of
this paper.

\subsubsection{One Particle in 2D and 3D Harmonic Oscillator}

\begin{figure}[H]
	\includegraphics[]{figures/one_part_training2.pdf}
	\caption{Energy estimated from each batch during training of the DNN model on one particle in harmonic oscillator, in $2D$ and $3D$, respectively}
	\label{fig:one_part_training2}
\end{figure}

The minimization of $\langle E \rangle$ during training of the DNN model of
one particle in 2D and 3D harmonic oscillator can be seen in
\autoref{fig:one_part_training1}. Both energies approach the correct ground
state energy, $E=1$ and $E=1.5$ in two and three dimensions, respectively. The
fluctuations in the energies decrease as the training progresses, indicating that the trial wave function approaches
the correct ground state, as $\sigma^2 \to 0$ when $\psi_{\text{Trial}} \to \psi_{\text{GS}}$.   


\begin{figure}[H]
	\includegraphics[]{figures/one_part_2D_dens.pdf}
	\caption{Radial one-body density for one particle in 2D harmonic oscillator.
      The density was calculated using $N=\num{1e6}$ samples from the trained DNN
      model, using $100$ bins on the interval $[0,3]$. It is compared to the
      analytical result} 
	\label{fig:one_part_2D_dens}
\end{figure}

\begin{figure}[H]
	\includegraphics[]{figures/one_part_3D_dens.pdf}
	\caption{Radial one-body density for one particle in 3D harmonic oscillator.
      The density was calculated using $N=\num{1e6}$ samples from the trained DNN
      model, using $100$ bins on the interval $[0,3]$. It is compared to the
      analytical result} 
	\label{fig:one_part_3D_dens}
\end{figure}

After the previous training, the radial one-body density is calculated using
$N=\num{1e6}$ samples generated by the model. The densities are presented in
\autoref{fig:one_part_2D_dens} and \autoref{fig:one_part_3D_dens}. Although the
approximation is not as close as seen in \autoref{fig:one_part_func}, it is
still fairly good.  

\subsubsection{Ground State Energies}

\begin{table}[ht]
	\begin{tabular}{l|SS}
		\toprule
		           & {Numerical} & {Analytical} \\
		1D, \(\tanh\)   & 0.5015(2) &   0.5      \\
		1D, ReLU   & 0.0032(2) &   0.5      \\
		2D, \(\tanh\)   & 1.0015(1) &   1.0        \\
		3D, \(\tanh\)   & 1.5046(1) &   1.5      \\
		\bottomrule
	\end{tabular}
	\caption{Summary of the estimated ground state energies of the DNN model
      trained on the previously discussed systems. The energy was estimated
      using $N=\num{1e6}$ samples} 
	\label{tab:1}
\end{table}

In \autoref{tab:1}, a summary of the estimated ground stated energies of the DNN
model trained on the previously discussed systems is presented. Without too much
concern for choice of parameters,all models but the ReLU model produces results
to accurate to within $1\%-3\%$ of the analytical values. 


\subsection{RNN-DNN Hybrid Model for Non-Interacting Bosonic Quantum Dots}
\begin{figure}[H]
	\includegraphics[]{figures/many_part_nonint_training1.pdf}
	\caption{Estimate of energy during training of RNN-DNN hybrid models. The models were trained on two non-interacting bosons in 2D and 3D harmonic oscillator. The models were trained for 1000 epochs, using a batch size of 500.}
	\label{fig:many_part_nonint_training1}
\end{figure}

\begin{figure}[H]
	\includegraphics[]{figures/many_part_nonint_training2.pdf}
	\caption{Estimate of energy during training of RNN-DNN hybrid models. The models were trained on six non-interacting bosons in 2D and 3D harmonic oscillator. The models were trained for 1000 epochs, using a batch size of 500.}
	\label{fig:many_part_nonint_training2}
\end{figure}

In \autoref{fig:many_part_nonint_training1} and
\autoref{fig:many_part_nonint_training2}, we see the minimization of energy as
the RNN-DNN model is trained on various numbers of non-interacting bosons in 2D
and 3D harmonic oscillators. In all cases, the model used a hidden state of 5
units, together with a two layer network with 32 nodes each. Although the
RNN-DNN uses the hidden state to encode the correlation with previously sampled
particles, it has no problem learning the ground state of uncorrelated systems,
as seen from the figures. Note that in
\autoref{fig:many_part_nonint_training2}, a sudden dip in the energy in the
beginning training can be seen for $N=6$ in 3D. This dip violates the
variational principle. As nothing was done to regularize the trial wave
function, it is perhaps unnormalizable during early stages of training, since
the initial parameters are random. This will likely cause the metropolis
algorithm to fail, as it is not possible to sample from such a distribution.
Nevertheless, the model approaches the correct ground state energy. 

\begin{table}[ht]
	\begin{tabular}{l|SS}
		\toprule
		& {Numerical}& {Analytical}     \\
		N=2, 2D    & 2.0017(5) &  2 \\
		N=6, 2D    & 6.0044(8) &  6 \\
		N=2, 3D    & 3.0021(5) &  3 \\
		N=6, 3D    & 9.012(1)  &  9 \\
		\bottomrule
	\end{tabular}
	\caption{Summary of the estimated ground state energies of the RNN-DNN model
      in the non-interacting case. The energy was estimated using $N=\num{1e5}$
      samples} 
	\label{tab:2}
\end{table}

From \autoref{tab:2}, we see that the RNN-DNN model attains good accuracy for the non-interacting case, both in 2D and 3D, averaging a relative error of $0.1\%$.

\subsection{RNN-DNN Model for Two Interacting Bosonic in 1D Quantum Dots}
\subsubsection{Training}

\begin{figure}[H]
	\includegraphics[]{figures/many_part_int_training1.pdf}
	\caption{Estimate of energy during training of RNN-DNN hybrid model. The
      model was trained on two interacting bosons in 1D harmonic oscillator. For
      the Coloumb interaction, a strength value of $\alpha = 1$ and a shielding
      value of $\beta = 0.1$ was used. The models were trained for 2000 epochs,
      using a batch size of 2000.} 
	\label{fig:many_part_int_training1}
\end{figure}

In \autoref{fig:many_part_int_training1}, we see the training process of two interacting bosons in 1D harmonic oscillator. The minimization of the energy is a bit more noisy than the previous non-interacting cases, hence the batch size was increased. The model used a hidden state of 5 units, together with a two layer network with 64 and 32 nodes, respectively.

Using $N=\num{1e6}$ samples, the ground state energy was estimated to be $E = 2.552(1)$. Compared to $E = 2.5482$ CISD energy using 40 orbitals(courtesy Øyvind Schøyen), the relative error is about $0.1\%$.

\subsubsection{One-Body Density}

\begin{figure}[H]
	\includegraphics[]{figures/many_part_int_onebody.pdf}
	\caption{One-body density for two interacting bosons in 1D harmonic
      oscillator. The density was calculated using $N=\num{4e5}$ samples and $100$
      bins on the interval $[-8,8]$.} 
	\label{fig:many_part_int_onebody}
\end{figure}

\begin{figure}[H]
	\includegraphics[scale = 0.4]{figures/oyvind.png}
	\caption{One-body density for two interacting bosons in 1D harmonic oscillator using CISD,courtesy Øyvind Schøyen}
	\label{fig:many_part_int_onebody}
\end{figure}

\autoref{fig:many_part_int_onebody} shows the corresponding one body density,
which shows the characteristic separation of two peaks caused by the repulsive
interaction. Comparing with the one-body density produced with the CISD
simulation, the peaks are not as sharply defined. In terms of ground state
energy, the RNN-DNN model successfully accounts for the correlation of the
system, showing that the hidden state is able to encode the correlation in a
meaningful way. However, like other approches in machine learning VMC, such as
RBM, it struggles to reproduce the correct one-body density.  

\subsubsection{Conditional Probabilities}

\begin{figure}[H]
	\includegraphics[]{figures/many_part_con1.pdf}
	\caption{One-body density for two interacting bosons in 1D harmonic
      oscillator. The density was calculated using $N=\num{4e5}$ samples and $100$
      bins on the interval $[-8,8]$.} 
	\label{fig:many_part_con1}
\end{figure}

\begin{figure}[H]
	\includegraphics[]{figures/many_part_con2.pdf}
	\caption{One-body density for two interacting bosons in 1D harmonic
      oscillator. The density was calculated using $N=\num{4e5}$ samples and $100$
      bins on the interval $[-8,8]$.} 
	\label{fig:many_part_con2}
\end{figure}

In \autoref{fig:many_part_con1} and \autoref{fig:many_part_con2}, we see the
conditional probabilities produced by the RNN-DNN model trained on two
interacting bosons in 1D harmonic oscillator. In \autoref{fig:many_part_con1},
we see that the probability density for particle 1 is skewed towards the left.
As there is no physical reason for why the first sampled particle is found more
often to the left, one should think of the conditional probabilities more as
rules for sampling the particles, rather than physical objects. In fact, given
$P(x_1, x_2)$, there is no unique way of factoring it to conditional
probabilities:  
\begin{align*}
	P(x_1, x_2) &= P(x_1)P(x_2|x_1)\\
                &= f(x_1)P(x_1)\frac{P(x_2|x_1)}{f(x_1)}\\
                &= \tilde{P}(x_1)\tilde{P}(x_2|x_1),
\end{align*}
where $f(x_1)$ is an arbitrary function. 

In \autoref{fig:many_part_con2}, we see however that some physical meaning is
retained. The probability density of \(x_{2}\) conditioned on \(x_{1}\)
acts as to avoid the probability density of \(x_{1}\). Given \(x_{1}\), the
probability of sampling \(x_{2}\) close to \(x_{1}\) is relatively small
compared to sampling it further away. This is a reflection of the repulsive
potential, which works to push the particles apart. 
 
\subsubsection{Sampling from Conditional Probabilities}

\begin{figure}[H]
	\includegraphics[]{figures/many_part_met.pdf}
	\caption{One-body density for two interacting bosons in 1D harmonic
      oscillator. The density was calculated using $N=\num{4e5}$ samples and $100$
      bins on the interval $[-8,8]$.} 
	\label{fig:many_part_met}
\end{figure}

Next, we examine if Brute force Metropolis sampling manages to 
sufficiently sample the distributions.
In \autoref{fig:many_part_met}, the target probability density is
$P(x_2|x_1=0.5)$ produced by the RNN-DNN model trained on the system as
previously discussed. The densities were estimated using Bruteforce Metropolis
with a step length of $1$ and varying number of thermalization steps. For a
low number of steps, such as $10$, it can be seen from the figure that the
estimated density does not match the one produced by the model. Since the random
walkers start off uniformly, they need a sufficient amount of steps so that they
don't end up getting stuck in low probability areas, as seen to the left in the
figure. If this happens, we fail to sample variables that are distributed as our
model dictates. This is remedied by taking ample amount of steps.
\subsection{RNN-DNN Model for Two Interacting Bosons In Higher Dimensions}

\subsubsection{Ground State Energy}

\begin{table}[ht]
	\begin{tabular}{r|SS}
		\toprule
		      & {RNN-DNN}   & {DMC}     \\
		2D    & 3.014(3)  &  3.00000(1) \\
		3D    & 3.751(1)  &   3.730123(3) \\
		\bottomrule
	\end{tabular}
	\caption{Estimated ground state energies of the RNN-DNN model train on two interacting bosons in 2D and 3D harmonic oscillator, respectively. In both cases, 10 hidden units and two layers of 64 and 32 nodes were used. The batch size was 4000 and the models were trained for 1000 epochs. The energy was estimated using $N=\num{1e5}$ samples.}
	\label{tab:3}
\end{table}

From \autoref{tab:3}, we see the estimated ground state energy of the RNN-DNN model trained on two interacting bosons in 2D and 3D harmonic oscillator. Taking the DMC calculation as ground truth, the numerical error is under $1\%$ in 2D and 3D, as with the 1D case. 

\subsubsection{One-Body Density}
\begin{figure}[H]
	\includegraphics[]{figures/many_part_3D.pdf}
	\caption{One-body density for two interacting bosons in 1D harmonic
      oscillator. The density was calculated using $N=\num{4e5}$ samples and $100$
      bins on the interval $[-8,8]$.} 
	\label{fig:many_part_3D}
\end{figure}

\begin{figure}[H]
	\includegraphics[scale=0.6]{figures/onebody3d.png}
	\caption{One-body density for two interacting bosons in 1D harmonic
      oscillator. The density was calculated using $N=\num{4e5}$ samples and $100$
      bins on the interval $[-8,8]$.} 
	\label{fig:many_part_DMC}
\end{figure}

In \autoref{fig:many_part_3D} and \autoref{fig:many_part_DMC} we see the radial
one-body density of two interacting bosons in 3D harmonic oscillator calculated
using our RNN-DNN model and DMC, respectably. Note that technically, the DMC
calculation was performed on a two electron system, but since this state is
symmetrical with respect to spatial coordinates, it is comparable to our bosonic
system. In addition, the density is not scaled. 

Since the figures are produced independently, they are hard to compare. However,
some general features can be seen: 1/3 of the max amplitude occur around $r=1$,
and the density have almost completely diminished around $r=2$. On the other
hand, the density produced by the RNN-DNN model is not as linear towards the
origin as the DMC density, again indicating the our model may struggle
reproducing the correct one-body density. 


%%% Local Variables:
%%% mode: latex
%%% TeX-master: "../main"
%%% End:
