\section{Method}\label{sec:Method}
\subsection{Architecture}


\subsection{Metropolis Sampling}


\subsection{Automatic Differentiation in Tensorflow}


\subsection{Optimization}


\subsection{Statistical Treatment}


\subsection{Experimental Setup}


\subsection{One-Body Density}
To extract a one-body density from a Monte-Carlo simulation, we partition space into a number of bins in an appropriate range where the wave function is large. The bin size can be chosen small to get finer details of the density, but will require more data to mitigate statistical error.

For each particle configuration produced at every Metropolis step, the number of particles coinciding with each bin is checked. The one-body density is then produced by averaging over all Metropolis steps.


%%% Local Variables:
%%% mode: latex
%%% TeX-master: "../main"
%%% End:
