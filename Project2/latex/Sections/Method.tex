\section{Method}\label{sec:Method}
\subsection{Architecture}


\subsection{Metropolis Sampling}


\subsection{Automatic Differentiation in Tensorflow}
Since we are studying Hamiltonians defined on continuous space, calculating the Laplacian of the trial wave function is necessary to account for the kinetic energy of the system. Since the RNN-DNN model is made using the popular neural network framework Tensorflow(2), it feels natural to use its automatic differentiation functionality to calculate the Laplacian of the model with respect to the inputs(the positions). However, for more than one dimensional input, it turns out to be quite difficult, although a work-around proposed by (sitat) resolved the difficulty. The problem was as follow: 

Starting with an input $x$ of dimension $(batch\_size, d)$, a forward pass through the model results in an output $y$ of dimension $(batch\_size, 1)$, since the wave function evaluates to a scalar value. When performing \textbf{grad\_y = tf.gradients(y,x)}, we get 
\begin{equation*}
	grad\_y = \left(\frac{\partial y}{\partial x_1}, \cdots, \frac{\partial y}{\partial x_d}\right),
\end{equation*}
which has dimensional $(batch\_size, d)$. The problem arises when calculating the second derivative \textbf{grad2\_y = tf.gradients(grad\_y,x)}

\begin{equation*}
	grad2\_y = \left(\sum_{i=1}^{d}\frac{\partial^2 y}{\partial x_1 \partial x_i}, \cdots, \sum_{i=1}^{d}\frac{\partial^2 y}{\partial x_1 \partial x_i}\right).
\end{equation*}
Since \textbf{tf.gradients} is meant to be used for gradient descent, a summation of the derivatives of all outputs w.r.t the input $x$ is performed, which should not be present in the Laplacian.
The naïve solution is to slice the output after the first differentiation, such as\newline 
\textbf{tf.gradients(grad\_y[i],x[i])}. Sadly, slicing this way does not respect the computational graph constructed during forward pass, and \textbf{x[i]} is interpreted as a new variable not related to \textbf{grad\_y[i]}. The fix is to use clever reshaping of the input variable \textbf{x} before passing it to the model, in order to allow for the wanted slicing without destroying the computational graph. Details can be seen in (sitat) and in our source code. 

\subsection{Optimization}
During the training procedure of the RNN-DNN model, we use the following expression for the loss value:
\begin{equation*}
	L = 2\sum_{i=1}^{N_B}\ln(\psi_i) (E_i - E)
\end{equation*}
where $N_B$ is the batch size, $psi_i$ and $E_i$ is the wave function and local energy of sample $i$ in the batch, and $E$ is the average local energy over the whole batch. During gradient decent, the derivative of the loss value is differentiated w.r.t. the model parameters, and the gradient describes (ref) is recovered.

During vanilla gradient decent, the parameters are updated by performing a small step in the opposite direction of the computed gradient

\begin{equation*}
	\theta \rightarrow \theta - \mu \nabla_\theta \langle E \rangle,
\end{equation*}
where $\theta$ is the parameters of the model, $\mu$ is the learning rate, and $\nabla_\theta \langle E \rangle$ is the derivative of the loss value. 

As a better alternative to vanilla gradient decent, we use Tensorflows ADAM optimizer with default values, as it is a popular choice in the machine learning community.  

\subsection{One-Body Density}
To extract a one-body density from a Monte-Carlo simulation, we partition space into a number of bins in an appropriate range where the wave function is large. The bin size can be chosen small to get finer details of the density, but will require more data to mitigate statistical error.

For each particle configuration produced at every Metropolis step, the number of particles coinciding with each bin is checked. The one-body density is then produced by averaging over all Metropolis steps.


%%% Local Variables:
%%% mode: latex
%%% TeX-master: "../main"
%%% End:
